% !TEX TS-program = pdflatex
% !TEX encoding = UTF-8 Unicode

% This is a simple template for a LaTeX document using the "article" class.
% See "book", "report", "letter" for other types of document.

\documentclass[11pt]{article} % use larger type; default would be 10pt

\usepackage[utf8]{inputenc} % set input encoding (not needed with XeLaTeX)

%%% Examples of Article customizations
% These packages are optional, depending whether you want the features they provide.
% See the LaTeX Companion or other references for full information.

%%% PAGE DIMENSIONS
\usepackage{geometry} % to change the page dimensions
\geometry{a4paper} % or letterpaper (US) or a5paper or....
% \geometry{margin=2in} % for example, change the margins to 2 inches all round
% \geometry{landscape} % set up the page for landscape
%   read geometry.pdf for detailed page layout information

\usepackage{graphicx} % support the \includegraphics command and options

% \usepackage[parfill]{parskip} % Activate to begin paragraphs with an empty line rather than an indent

%%% PACKAGES
\usepackage{booktabs} % for much better looking tables
\usepackage{array} % for better arrays (eg matrices) in maths
\usepackage{paralist} % very flexible & customisable lists (eg. enumerate/itemize, etc.)
\usepackage{verbatim} % adds environment for commenting out blocks of text & for better verbatim
\usepackage{subfig} % make it possible to include more than one captioned figure/table in a single float
% These packages are all incorporated in the memoir class to one degree or another...
\usepackage{listings}

\title{Homework 1}
\author{Chenyue Wu}
%\date{} % Activate to display a given date or no date (if empty),
         % otherwise the current date is printed 

\begin{document}
\maketitle

\section{Problem1}

\subsection{Test the program for a large N}

On my local machine:

\begin{lstlisting}
$ mpirun -np 4 ./int_ring 10000000
After 10000000 loops on 4 processes, the integer grows to 60000000.
Total communication: 40000000.
Time elapsed: 20.095609 seconds.
Latency: 0.000001 seconds.
\end{lstlisting}

On remote machine:
\begin{lstlisting}
$ mpirun -np 4 -hosts crunchy1,crunchy3 ./int_ring 10000
After 10000 loops on 4 processes, the integer grows to 60000.
Total communication: 40000.
Time elapsed: 2.303186 seconds.
Latency: 0.000058 seconds.

\end{lstlisting}

\subsection{Array of 2MB}
On my local machine

\begin{lstlisting}
mpirun -np 4 ./array_ring.o 1000
Total communication: 4000.
Time elapsed: 2.633973 seconds.
Latency: 0.000658 seconds.
Bandwidth 3.037236GB/s
\end{lstlisting}

\section{Problem2}

\subsection{Result is independent of p}
I save u to different txt files and use 'cmp' to compare the files. It shows that result is independent of p.

\subsection{Strong scaling}
It is not strong scaling.

\begin{lstlisting}
$ mpirun -np 2 ./jacobi-mpi 100000 10
Results saved in vec100000loop10np2.txt.
Time elapsed: 0.058185 seconds.
$ mpirun -np 4 ./jacobi-mpi 100000 10
Results saved in vec100000loop10np4.txt.
Time elapsed: 0.085044 seconds.
$ mpirun -np 8 ./jacobi-mpi 100000 10
Results saved in vec100000loop10np8.txt.
Time elapsed: 0.484799 seconds.
\end{lstlisting}

\subsection{Parallel version of the Gauss-Seidel smoother}
Parallel version of the Gauss-Seidel smoother is more difficult because u needs to be sequentially updated in each step.

\end{document}
